% Version control information:
%$HeadURL: https://practicas-spss.googlecode.com/svn/trunk/correlacion/correlacion.tex $
%$LastChangedDate: 2010-09-27 16:37:11 +0200 (lun, 27 sep 2010) $
%$LastChangedRevision: 3 $
%$LastChangedBy: asalber $
%$Id: correlacion.tex 3 2010-09-27 14:37:11Z asalber $

\chapter{Correlación}

\section{Fundamentos teóricos}
El principal objetivo de la regresión simple es construir un modelo funcional
$y=f(x)$ que explique lo mejor posible la relación entre dos variables $X$
(variable independiente) e $Y$ (variable dependiente) medidas en una misma
muestra. Generalmente, el modelo construido se utiliza para realizar
inferencias predictivas de $Y$ en función de $X$ en el resto de la población.
Pero aunque la regresión garantiza que el modelo construido es el mejor
posible, dentro del tipo de modelo elegido (lineal, polinómico, exponencial,
logarítmico, etc.), puede que aún así, no sea un buen modelo para hacer
predicciones, precisamente porque no haya relación de ese tipo entre $X$ e
$Y$. Así pues, con el fin de validar un modelo para realizar predicciones
fiables, se necesitan medidas que nos hablen del grado de dependencia entre $X$ e
$Y$, con respecto a un modelo de regresión construido. Estas medidas se conocen
como medidas de \emph{correlación}.

Dependiendo del tipo de modelo ajustado, habrá distintos tipos de medidas de
correlación. Así, si el modelo de regresión construido es una recta,
hablaremos de correlación lineal; si es un polinomio, hablaremos de correlación
polinómica; si es una función exponencial, hablaremos de correlación
exponencial, etc. En cualquier caso, estas medidas nos hablarán de lo bueno
que es el modelo construido, y como consecuencia, de si podemos fiarnos de las
predicciones realizadas con dicho modelo.

La mayoría de las medidas de correlación surgen del estudio de los residuos o
errores en $Y$, que son las distancias de los puntos del diagrama de
dispersión a la curva de regresión construida, medidas en el eje $Y$, tal y
como se muestra en la figura ~(\ref{g:residuos2}). Estas distancias, son en
realidad, los errores predictivos del modelo sobre los propios valores de la
muestra.

\begin{figure}[h!]
  \centering
  \scalebox{0.8}{\input{correlacion/img/residuos_y}}
  \caption{Residuos o errores en $Y$. El residuo correspondiente a un punto $(x_i,y_j)$
  es la diferencia entre el valor $y_j$ observado en la muestra, y el valor
  teórico del modelo $f(x_i)$, es decir, $e_{ij}=y_j-f(x_i)$.}\label{g:residuos2}
\end{figure}

Cuanto más pequeños sean los residuos, mejor se ajustará el modelo a la nube
de puntos, y por tanto, mejor explicará la relación entre $X$ e $Y$. Cuando
todos los residuos son nulos, la curva de regresión pasa por todos los puntos
de la nube, y entonces se dice que la relación es perfecta, o bien que existe
una dependencia funcional entre $X$ e $Y$ (figura~\ref{g:dependenciafuncional2}).
Por contra, cuando los residuos sean grandes, el modelo no explicará bien la
relación entre $X$ e $Y$, y por tanto, sus predicciones no serán fiables
(figura~\ref{g:independencialineal2}).

\begin{figure}[h!]
\centering \subfigure[Dependencia funcional lineal.]
{\label{g:dependenciafuncional2}
\scalebox{0.7}{\input{correlacion/img/rectas_dependencia_lineal_perfecta}}}\qquad
\subfigure[Independencia
lineal.]{\label{g:independencialineal2}
\scalebox{0.7}{\input{correlacion/img/rectas_independencia_lineal}}}
\caption{Distintos grados de dependencia. En el primer caso, la relación es
perfecta y los residuos son nulos. En el segundo caso no existe relación
lineal y los residuos son grandes.}
\end{figure}

\subsection{Varianza residual}
Una primera medida de correlación, construida a partir de los residuos es la
\emph{varianza residual}, que se define como el promedio de los residuos al
cuadrado:
\[
s^2_{ry}=\frac{\sum_{i,j} e_{ij}^2 n_{ij}}{n}= \frac{\sum_{i,j} (y_j-f(x_i))^2
n_{ij}}{n}.
\]

Cuando los residuos son nulos, entonces $s^2_{ry}=0$ y eso indica que hay
dependencia funcional. Por otro lado, cuando las variables son independientes,
con respecto al modelo de regresión ajustado, entonces los residuos se
convierten en las desviaciones de los valores de $Y$ con respecto a su media, y se cumple
que $s^2_{ry}=s_y^2$. Así pues, se cumple que
\[  0 \leq s^2_{ry}\leq s_y^2. \]
Según esto, cuanto menor sea la varianza residual, mayor será la dependencia
entre $X$ e $Y$, de acuerdo al modelo ajustado. No obstante, la varianza tiene
como unidades las unidades de $Y$ al cuadrado, y eso dificulta su
interpretación.

\subsection{Coeficiente de determinación}
Puesto que el valor máximo que puede tomar la varianza residual es la varianza
de $Y$, se puede definir fácilmente un coeficiente a partir de la comparación
de ambas medidas. Surge así el \emph{coeficiente de determinación} que se
define como
\[
R^2=1-\frac{s^2_{ry}}{s_y^2}.
\]

Se cumple que
\[ 0\leq R^2\leq 1,\]
y además no tiene unidades, por lo que es más fácil de interpretar que la
varianza residual:
\begin{itemize}
\item $R^2=0$ indica que existe independencia según el
tipo de relación planteada por el modelo de regresión.
\item $R^2=1$ indica dependencia funcional.
\end{itemize}
Por tanto, cuanto mayor sea $R^2$, mejor será el modelo de regresión.

Si multiplicamos el coeficiente de determinación por 100, se obtiene el
porcentaje de variabilidad de $Y$ que explica el modelo de regresión. El
porcentaje restante corresponde a la variabilidad que queda por explicar y se
corresponde con el error predictivo del modelo. Así, por ejemplo, si tenemos
un coeficiente de determinación $R^2=0.5$, el modelo de regresión explicaría
la mitad de la variabilidad de $Y$, y en consecuencia, si se utiliza dicho
modelo para hacer predicciones, estas tendrían la mitad de error que si no se
utilizase, y se tomase como valor de la predicción el valor de la media de $Y$.

\subsubsection{Coeficiente de determinación lineal}
En el caso de que el modelo de regresión sea lineal, la fórmula del
coeficiente de determinación se simplifica y se convierte en
\[
r^2=\frac{s_{xy}^2}{s_x^2 s_y^2},
\]
que se conoce como \emph{coeficiente de determinación lineal}.

\subsection{Coeficiente de correlación}
Otra medida de dependencia bastante habitual es el \emph{coeficiente de
correlación}, que se define como la raíz cuadrada del coeficiente de
determinación:
\[
R=\pm\sqrt{1-\frac{s^2_{ry}}{s_y^2}},
\]
tomando la raíz del mismo signo que la covarianza.

La única ventaja del coeficiente de correlación con respecto al coeficiente de
determinación, es que tiene signo, y por tanto, además del grado de
dependencia entre $X$ e $Y$, también nos habla de si la relación es directa
(signo +) o inversa (signo -). Su interpretación es:
\begin{itemize}
\item $R=0$ indica independencia con respecto al tipo de relación planteada por el modelo de
regresión.
\item $R=-1$ indica dependencia funcional inversa.
\item $R=1$ indica dependencia funcional directa.
\end{itemize}
Por consiguiente, cuanto más próximo esté a -1 o a 1, mejor será el modelo de
regresión.

\subsubsection{Coeficiente de correlación lineal}
Al igual que ocurría con el coeficiente de determinación, cuando el modelo de
regresión es lineal, la fórmula del coeficiente de correlación se convierte en
\[
r=\frac{s_{xy}}{s_x s_y},
\]
y se llama \emph{coeficiente de correlación lineal}.

Por último, conviene remarcar que un coeficiente de determinación o de
correlación nulo, indica que hay independencia según el modelo de regresión
construido, pero puede haber dependencia de otro tipo. Esto se ve claramente en el ejemplo de  la figura~\ref{g:dependenciaparabolica}.

\begin{figure}[h!]
\centering \subfigure[Dependencia lineal débil.]
{\label{g:dependencialinealdebil}
\scalebox{0.7}{\input{correlacion/img/recta_regresion_relacion_parabolica}}}\qquad
\subfigure[Dependencia parabólica fuerte.] {\label{g:dependenciaparabolicafuerte}
\scalebox{0.7}{\input{correlacion/img/regresion_parabolica}}}
\caption{En la figura de la izquierda se ha ajustado un modelo lineal y se ha obtenido un
$R^2=0$, lo que indica que el modelo no explica nada de la relación entre $X$ e
$Y$, pero no podemos afirmar que $X$ e $Y$ son independientes. De hecho, en la
figura de la derecha se observa que al ajustar un modelo parabólico, $R^2=0.97$,
lo que indica que casi hay una dependencia funcional parabólica entre $X$ e $Y$.}
\label{g:dependenciaparabolica}
\end{figure}

\subsubsection{Fiabilidad de las predicciones}
Aunque el coeficiente de determinación o de correlación nos hablan de la bondad de un modelo de regresión, no es el único dato que hay que tener en cuenta a la hora de hacer predicciones.

La fiabilidad de las predicciones que hagamos con un modelo de regresión depende de varias cosas:
\begin{itemize}
\item El coeficiente de determinación: Cuando mayor sea, menores serán los errores predictivos y mayor la fiabilidad de las predicciones.
\item La variablidad de la población: Cuanto más variable es una población, más difícil es predecir y por tanto menos fiables serán las predicciones del modelo.
\item El tamaño muestral: Cuanto mayor sea, más información tendremos y, en consecuencia, más fiables serán las predicciones. 
\end{itemize} 

Además, hay que tener en cuenta que un modelo de regresión es válido para el
rango de valores observados en la muestra, pero fuera de ese rango no tenemos
información del tipo de relación entre las variables, por lo que no deberíamos
hacer predicciones para valores que estén lejos de los observados en la
muestra.


\section{Ejercicios prácticos}
\begin{enumerate}[leftmargin=*]

\item  En una licenciatura se quiere estudiar la relación entre el número
medio de horas de estudio diarias y el número de asignaturas suspensas. Para ello se
obtuvo la siguiente muestra:
\begin{center}
\begin{tabular}{cccccccc}
  Horas & Suspensos &  & Horas & Suspensos & & Horas & Suspensos  \\
  \cline{1-2}\cline{4-5}\cline{7-8}
  3.5 & 1 & & 2.2 & 2 & & 1.3 & 4 \\
  0.6 & 5 & & 3.3 & 0 & & 3.1 & 0 \\
  2.8 & 1 & & 1.7 & 3 & & 2.3 & 2 \\
  2.5 & 3 & & 1.1 & 3 & & 3.2 & 2 \\
  2.6 & 1 & & 2.0 & 3 & & 0.9 & 4 \\
  3.9 & 0 & & 3.5 & 0 & & 1.7 & 2 \\
  1.5 & 3 & & 2.1 & 2 & & 0.2 & 5 \\
  0.7 & 3 & & 1.8 & 2 & & 2.9 & 1 \\
  3.6 & 1 & & 1.1 & 4 & & 1.0 & 3 \\
  3.7 & 1 & & 0.7 & 4 & & 2.3 & 2 \\
\end{tabular}

\end{center}

Se pide:

\begin{enumerate}
\item  Crear las variables \textsf{horasestudio} y
\textsf{suspensos} e introducir estos datos.

\item Calcular los coeficientes de correlación y de determinación
lineal. ¿Es un buen modelo la recta de regresión? ¿Qué porcentaje
de la variabilidad del número de suspensos está explicada por el
modelo?

\begin{indicacion}{
\begin{enumerate}
\item Seleccionar el menú \menu{Analizar\flecha Regresion\flecha Lineal...}.

\item Seleccionar la variable \variable{suspensos} en el campo
\texttt{Dependiente} del cuadro de diálogo.

\item Seleccionar la variable \variable{horasestudio} en el campo
\texttt{Independiente} del cuadro de diálogo y hacer click sobre
el botón \boton{Aceptar}.

\item Observaremos en la ventana de resultados obtenida, la tabla
denominada \texttt{Coeficientes}, y en la columna \texttt{B} de
los \texttt{Resumen del modelo}, encontramos el valor del
coeficiente de correlación \texttt{R} y el del coeficiente de
determinación \texttt{R cuadrado}.


\end{enumerate}}
\end{indicacion}


\item Utilizar la recta de regresión para predecir el número de
suspensos correspondiente a 3 horas de estudio diarias. ¿Es fiable
esta predicción?

\begin{indicacion}{
\begin{enumerate}

\item Para escribir la recta, observaremos en la ventana de
resultados obtenida, la tabla denominada \texttt{Coeficientes}, y
en la columna \texttt{B} de los \texttt{Coeficientes no
estandarizados}, encontramos en la primera fila la
\textsf{constante} de la recta y en la segunda la
\texttt{pendiente}.

\item Crear una nueva variable \texttt{valores} e introducir los
valores que queremos estudiar.

\item Crear otra nueva variable \texttt{prediccion}, seleccionando
el  menú \menu{Transformar\flecha Calcular...}.

\item  Introducir el nombre de la nueva variable
\texttt{prediccion} en el campo \campo{Variable de destino} del
cuadro de diálogo.

\item Introducir la ecuación de la recta en el campo
\texttt{Expresión numérica}, utilizando los coeficientes citados
anteriormente  y la variable \texttt{valores} y hacer click sobre
el botón \boton{Aceptar}.
\end{enumerate}}
\end{indicacion}


\item Según el modelo lineal, ¿cuántas horas diarias tendrá que
estudiar como mínimo un alumno si quiere aprobarlo todo?.

\begin{indicacion}{
Seguir los mismos pasos de los apartados anteriores, pero
escogiendo como variable dependiente \textsf{horas estudio}, y
como independiente \textsf{suspensos}}
\end{indicacion}



\end{enumerate}

\item Después de tomar un litro de vino se ha medido la
concentración de alcohol en la sangre en distintos instantes, obteniendo:
\[
\begin{tabular}{|c|c|c|c|c|c|c|c|}
\hline Tiempo después (minutos) & 30 & 60 & 90 & 120 & 150 & 180 & 210\\ \hline
Concentración (gramos/litro) & 1.6 & 1.7 & 1.5 & 1.1 & 0.7 & 0.2 & 2.1\\
\hline
\end{tabular}
\]
Se pide:

\begin{enumerate}
\item Crear las variables \textsf{tiempo} y \textsf{alcohol} e introducir estos datos.

\item Calcular el coeficiente de correlación lineal e
interpretarlo.

\begin{indicacion}{
\begin{enumerate}
\item Seleccionar el menú
\texttt{Analizar\flecha Correlaciones\flecha Bivariadas...}.

\item Seleccionar ambas variables en el campo \campo{Variables}
del cuadro de diálogo y hacer click sobre el botón
\texttt{Aceptar}.
\end{enumerate}}
\end{indicacion}

\item  Dibujar el diagrama de dispersión junto con la recta
ajustada correspondiente a \texttt{alcohol} sobre \texttt{tiempo}.
¿Existe algún individuo con un residuo demasiado grande? Si es
así, eliminar dicho individuo de la muestra y volver a calcular el
coeficiente de correlación. ¿Ha mejorado el modelo?

\begin{indicacion}{
\begin{enumerate}
\item Seleccionar el menú \menu{Gráficos\flecha Dispersión...}, elegir
la opción \opcion{simple} y  hacer click sobre el botón
\texttt{Definir}. \item Seleccionar la variable \variable{alcohol}
en el campo \campo{Eje Y} del cuadro de diálogo. \item
Seleccionar la variable \variable{tiempo} en el campo \campo{Eje X}
del cuadro de diálogo y hacer click sobre el botón
\texttt{Aceptar}.

\item Editar el gráfico realizado anteriormente haciendo un doble
click sobre él.

\item Seleccionar los puntos haciendo click sobre alguno de ellos.
\item Seleccionar el menú \menu{Gráfico\flecha Añadir elemento de
gráfico\flecha Linea de ajuste total} (También se podría usar en lugar
del menu, la barra de herramientas) \item Cerrar el editor de
gráficos, cerrando la ventana.

\item Si existe algún individuo con un residuo demasiado grande,
ir a la ventana del \texttt{Editor de datos}, y eliminarlo.

\item Repetir los pasos del apartado anterior.
\end{enumerate}}
\end{indicacion}


\item  Si la concentración máxima de alcohol en la sangre que
permite la ley para poder conducir es 0.4 g/l, ¿cuánto tiempo
habrá que esperar después de tomarse un litro de vino para poder
conducir sin infringir la ley? ¿Es fiable esta predicción?

\begin{indicacion}{
\begin{enumerate}
\item Seleccionar el menú \menu{Analizar\flecha Regresión\flecha Lineal...}.

\item Seleccionar la variable \variable{tiempo} en el campo
\texttt{Dependiente} del cuadro de diálogo.

\item Seleccionar la variable \variable{alcohol} en el campo
\texttt{Independiente} del cuadro de diálogo y hacer click sobre
el botón \boton{Aceptar}.

\item Para escribir la recta, observaremos en la ventana de
resultados obtenida, la tabla denominada \texttt{Coeficientes}, y
en la columna \texttt{B} de los \texttt{Coeficientes no
estandarizados}, encontramos en la primera fila la
\textsf{constante} de la recta y en la segunda la
\texttt{pendiente}.

\item Crear una nueva variable \texttt{valores} e introducir los
valores que queremos estudiar.

\item Crear otra nueva variable \texttt{prediccion}, seleccionando
el  menú \menu{Transformar\flecha Calcular...}.

\item  Introducir el nombre de la nueva variable
\texttt{prediccion} en el campo \campo{Variable de destino} del
cuadro de diálogo.

\item Introducir la ecuación de la recta en el campo
\texttt{Expresión numérica}, utilizando los coeficientes citados
anteriormente  y la variable \texttt{valores} y hacer click sobre
el botón \boton{Aceptar}.

\end{enumerate}}
\end{indicacion}



\end{enumerate}

\end{enumerate}


\section{Problemas}
\begin{enumerate}[leftmargin=*]
\item  Se determina la pérdida de actividad que experimenta un
medicamento desde el momento de su fabricación a lo largo del tiempo, obteniéndose
el siguiente resultado:

\begin{center}
\begin{tabular}{|c|c|c|c|c|c|}
\hline Tiempo (en años) & 1 & 2 & 3 & 4 & 5 \\ \hline Actividad restante (\%) & 96 &
84 & 70 & 58 & 52 \\ \hline
\end{tabular}
\end{center}

Se desea calcular:

\begin{enumerate}
\item  Calcular el coeficiente de determinación e interpretarlo.

\item ¿Cuándo tiempo debe pasar para que el fármaco tenga una actividad del 80\%? ¿Cuándo será nula la actividad? ¿Son igualmente fiables estas predicciones?
\end{enumerate}

\item Al realizar un estudio sobre la dosificación de un cierto
medicamento, se trataron 6 pacientes con dosis diarias de 2 mg, 7 pacientes con 3 mg
y otros 7 pacientes con 4 mg. De los pacientes tratados con 2 mg, 2 curaron al cabo
de 5 días, y 4 al cabo de 6 días. De los pacientes tratados con 3 mg diarios, 2
curaron al cabo de 3 días, 4 al cabo de 5 días y 1 al cabo de 6 días. Y de los
pacientes tratados con 4 mg diarios, 5 curaron al cabo de 3 días y 2 al cabo de 5
días. Se pide:

\begin{enumerate}
\item  Calcular el coeficiente de correlación lineal e interpretarlo.

\item  Determinar el tiempo esperado de curación para una dosis de 5 mg diarios. ¿Es fiable esta predicción?

\end{enumerate}

\end{enumerate}
