% Version control information:
%$HeadURL: https://practicas-spss.googlecode.com/svn/trunk/anova_1_factor/anova_1_factor.tex $
%$LastChangedDate: 2010-09-27 16:37:11 +0200 (lun, 27 sep 2010) $
%$LastChangedRevision: 3 $
%$LastChangedBy: asalber $
%$Id: anova_1_factor.tex 3 2010-09-27 14:37:11Z asalber $

\chapter{Contrastes Basados en el Estadístico $\chi^{2}$}
\section{Fundamentos teóricos}
Existen multitud de situaciones en el ámbito de la salud, o en 
cualquier otro ámbito, en las que el investigador está interesado en determinar posibles relaciones entre variables cualitativas. Un ejemplo podría ser el estudio de si existe relación entre las complicaciones tras una intervención quirúrgica y el
 sexo del paciente, o bien el hospital en el que se lleva a cabo 
la intervención.
En este caso, todas las técnicas de inferencia vistas hasta ahora para variables cuantitativas no son aplicables, y para ello utilizaremos un contraste de hipótesis basado en el estadístico $\chi^{2}$ (Chi-cuadrado).

Sin embargo, aunque éste sea su aspecto
 más conocido, el uso del test no se limita al estudio de la 
posible relación entre variables cualitativas, y también se aplica para comprobar el ajuste de la distribución muestral de una
 variable, ya sea cualitativa o cuantitativa, a su hipotético
modelo teórico de distribución.

En general, este tipo de tests consiste en tomar una muestra y
observar si hay diferencia significativa entre las \emph{frecuencias
observadas} y las especificadas por la ley teórica del modelo que
se contrasta, también denominadas \emph{frecuencias esperadas}.

Podríamos decir que existen dos grandes bloques de aplicaciones
básicas en el uso del test de la $\chi^{2}$:

\begin{enumerate}
\item \textbf{Test de ajuste de distribuciones}. Es un contraste
de significación para saber si los datos de la población, de la
cual hemos extraído una muestra, son conforme a una ley de
distribución teórica que sospechamos que es la correcta.

Por ejemplo: disponemos de 400 datos que, a priori, siguen una
distribución de probabilidad uniforme, pero ¿es estadísticamente
cierto que se ajusten a dicho tipo de distribución?


\item \textbf{Test para tablas de contingencia.} En las que se parte de la tabla de frecuencias bidimensional para las distintas modalidades de las variables cualitativas. Aunque muy a menudo el test de la
$\chi^{2}$ aplicado en tablas de contingencia se denomina prueba
de independencia, en realidad se aplica en dos diseños
experimentales diferentes, que hacen que se clasifique en dos
bloques diferentes:


\begin{enumerate}
\item \textbf{Prueba de independencia}. Mediante la que el
investigador pretende estudiar la relación entre dos variables cualitativas en una población.

Por ejemplo: tenemos una muestra de 200 enfermos (el investigador
tan sólo controla el total en una muestra) operados de apendicitis
en 4 hospitales diferentes y queremos ver si hay relación entre la
posible infección postoperatoria y el hospital en el que el paciente ha sido operado.

\item \textbf{Prueba de homogeneidad}. Mediante la que el
investigador pretende ver si la proporción de una determinada
característica es la misma en poblaciones, tal vez, diferentes.

Por ejemplo: tenemos dos muestras diferentes, una de ellas de 100
individuos VIH positivos, y otra de 600 VIH negativos (el
investigador controla el total en ambas muestras), y queremos
analizar si la proporción de individuos con problemas
gastrointestinales es la misma en ambas.
\end{enumerate}
\end{enumerate}

Por último, aunque el test de la Chi-cuadrado es muy importante en el análisis de las relaciones entre variables cualitativas, su aplicación puede conducir a errores en determinadas situaciones; sobre todo cuando los tamaños muestrales son pequeños, lo cual conduce a que en algunas categorías apenas tengamos individuos y ello invalida los supuestos de aplicación del test; y también cuando tenemos variables cualitativas analizadas en los mismos individuos pero en diferentes tiempos, es decir, mediante datos pareados. Para el primer caso, cuando el número de individuos en alguna categoría es muy pequeño, se utiliza el test Exacto de Fisher, mientras que en el segundo, con datos pareados, se utiliza el test de McNemar.


\subsection{Contraste $\chi^{2}$ de Pearson para ajuste de distribuciones}
Es el contrate de ajuste más antiguo y es válido para todo tipo de
distribuciones. Para analizar una muestra de una variable agrupada
en categorías (aunque sea cuantitativa), evaluando una hipótesis
previa sobre probabilidad de cada modalidad o categoría, se realiza un contraste de hipótesis Chi-cuadrado de bondad de ajuste.

El contraste se basa en hacer un recuento de los datos y comparar las
frecuencias observadas de cada una de las modalidades con las
frecuencias esperadas por el modelo teórico que se contrasta.
De este modo, se calcula el estadístico:
\[
\chi^2 = \sum_{i=1}^k \frac{(O_i-E_i)^2}{E_i},
\]
donde $O_i$ son las frecuencias observadas en la muestra en la
modalidad $i$, y  $E_i$ son las frecuencias esperadas para la
misma modalidad según el modelo teórico. Las frecuencias esperadas
se calculan multiplicando el tamaño de la muestra por la
probabilidad de la correspondiente modalidad según el modelo
teórico, es decir $E_i=np_i$, siendo $p_i$ la probabilidad de la modalidad $i$.

Si la población de la que se ha obtenido la muestra sigue el
modelo de distribución teórica, el estadístico anterior se
distribuye como $\chi^{2}$ con $k-1$ grados de libertad, donde $k$
es el número de modalidades de la variable. Un valor del
estadístico $\chi^{2}$ grande indica que las distribuciones de las
frecuencias observadas y esperadas son bastantes diferentes,
mientras que un valor pequeño del estadístico indica que hay poca
diferencia entre ellas.

La prueba $\chi^2$ de bondad del ajuste es válida si todas las
frecuencias esperadas son mayores o iguales que 1 y no más de un
$20\%$ de ellas tienen frecuencias esperadas menores que 5. Si no
se cumple lo anterior, entonces las categorías implicadas deben
combinarse con categorías adyacentes para garantizar que todas
cumplen la condición. Si las categorías corresponden a variables
cuantitativas categorizadas, no tienen necesariamente que
corresponder a la misma amplitud de variable.


\subsection{Contraste $\chi^{2}$ en tablas de contingencia}
Como ya hemos visto, el contraste de la $\chi^2$ en tablas de
contingencia sirve para establecer relaciones entre variables
cualitativas (o cuantitativas categorizadas), entre las que no
puede realizarse un análisis de regresión y correlación, y tanto
para determinar independencia entre variables, como homogeneidad
entre poblaciones (igual proporción de una determinada
característica). Para ello, describimos el proceso metodológico en
el caso de independencia entre variables, que en la práctica, y
aunque conceptualmente son casos diferentes, es el mismo también
para la homogeneidad entre poblaciones.

Por tablas de contingencia se entiende aquellas tablas de doble
entrada donde se realiza una clasificación de la muestra de
acuerdo a un doble criterio de clasificación. Por ejemplo, la
clasificación de unos individuos de acuerdo a su sexo y su grupo
sanguíneo crearía una tabla donde cada celda de la tabla
representaría la frecuencia bivariante de las características
correspondientes a su fila y columna (por ejemplo mujeres de grupo
sanguíneo A). Si se toma una muestra aleatoria de tamaño $n$ en la
que se miden ambas variables y se representan las frecuencias de
los pares observados en una tabla bidimensional, tenemos:

\[
\begin{tabular}{|l|lllll|l|}
\cline{1-6}
$X/Y$ & \multicolumn{1}{c}{} & \multicolumn{1}{c}{} & \multicolumn{1}{c}{$y_j$} & \multicolumn{1}{c}{} & \multicolumn{1}{c|}{} & \multicolumn{1}{l}{} \\
\hline
\multicolumn{1}{|c|}{} & \multicolumn{1}{c}{} & \multicolumn{1}{c}{} & \multicolumn{1}{c}{} & \multicolumn{1}{c}{} & \multicolumn{1}{c|}{} & \multicolumn{1}{c|}{} \\
\multicolumn{1}{|c|}{} & \multicolumn{1}{c}{} & \multicolumn{1}{c}{} & \multicolumn{1}{c}{} & \multicolumn{1}{c}{} & \multicolumn{1}{c|}{} & \multicolumn{1}{c|}{} \\
\multicolumn{1}{|c|}{$x_i$} & \multicolumn{1}{c}{} & \multicolumn{1}{c}{} & \multicolumn{1}{c}{$n_{ij}$} & \multicolumn{1}{c}{} & \multicolumn{1}{c|}{} & \multicolumn{1}{c|}{$n_i$} \\
\multicolumn{1}{|c|}{} & \multicolumn{1}{c}{} & \multicolumn{1}{c}{} & \multicolumn{1}{c}{} & \multicolumn{1}{c}{} & \multicolumn{1}{c|}{} & \multicolumn{1}{c|}{} \\
\multicolumn{1}{|c|}{} & \multicolumn{1}{c}{} & \multicolumn{1}{c}{} & \multicolumn{1}{c}{} & \multicolumn{1}{c}{} & \multicolumn{1}{c|}{} & \multicolumn{1}{c|}{} \\
\hline
\multicolumn{1}{c|}{} & \multicolumn{1}{c}{} & \multicolumn{1}{c}{} & \multicolumn{1}{c}{$n_j$} & \multicolumn{1}{c}{} & \multicolumn{1}{c|}{} & \multicolumn{1}{c|}{$n$} \\
\cline{2-7}
\end{tabular}
\]
Donde $n_{ij}$ es la frecuencia absoluta del par $(x_i, y_j)$,
$n_i$ es la frecuencia marginal de la modalidad $x_i$ y $n_j$ es
la frecuencia marginal de la modalidad $y_j$. Dichas frecuencias
aparecen en los márgenes de la tabla de contingencia sumando las
frecuencias por filas y columnas, y por ello se conocen como
frecuencias marginales.

Siguiendo un procedimiento parecido al del apartado anterior, se
comparan las frecuencias observadas en la muestra (frecuencias
reales) con las frecuencias esperadas (frecuencias teóricas). Para
ello, calculamos la probabilidad de cada casilla de la tabla
teniendo en cuenta que si ambas variables son independientes la
probabilidad de cada celda surge como un producto de
probabilidades (probabilidad de la intersección de dos sucesos
independientes) $p_{ij}=p_ip_j=\frac{n_i}{n}\frac{n_j}{n}$. De este modo,
obtenemos la frecuencia esperada como:
\[E_i=np_{ij}=n\frac{n_i}{n}\frac{n_j}{n}=\frac{n_in_j}{n},\]
Y con ello se calcula el estadístico de la Chi-cuadrado de Pearson:
\[
\chi^2 = \sum_{i,j}\frac{(O_{ij}-E_{ij})^2}{E_{ij}}.
\]
En el caso de que $X$ e $Y$ fuesen independientes, este
estadístico presenta una distribución Chi-cuadrado con
$(f-1)(c-1)$ grados de libertad, donde $f$ es el número de filas
de la tabla de contingencia y $c$ el número de columnas. Un valor
del estadístico Chi-cuadrado grande indica que las distribuciones
de las frecuencias observadas y esperadas son bastantes
diferentes, y por lo tanto falta de independencia; mientras que un
valor pequeño del estadístico indica que hay poca diferencia entre
ellas, lo cual nos indica que son independientes.

Este test es adecuado si las frecuencias esperadas para cada celda
valen como mínimo 1 y no más de un $20\%$ de ellas tienen
frecuencias esperadas menores que 5. En el caso de una tabla 2x2,
estas cifras se alcanzan sólo cuando ninguna frecuencia esperada
es menor que 5. Si esto no se cumple, puede, entre otras,
utilizarse una prueba para pequeñas muestras llamada prueba exacta
de Fisher.


\subsection{Test exacto de Fisher}
Este test se puede utilizar cuando no se cumplan las condiciones necesarias para aplicar el test de la Chi cuadrado (más de un 20\% de las frecuencias esperadas para cada celda son menores que 5). Aunque, dada la gran cantidad de cálculos necesarios para llegar al resultado final del test, los programas de Estadística sólo lo calculaban para tablas de contingencia 2x2, en versiones actuales pueden incorporar módulos específicos que amplían la capacidad del núcleo del programa base, y que sí que permiten evaluar el test de Fisher en tablas con más categorías. Por ejemplo, en PASW, si se ha instalado el módulo de Pruebas Exactas, se puede calcular el test Exacto de Fisher en tablas generales con F categorías en las filas y  C categorías en las columnas.

El test Exacto de Fisher está basado en la distribución exacta de los datos y no en aproximaciones asintóticas, y presupone que los marginales de la tabla de contingencia están fijos. El procedimiento para su cálculo consiste en evaluar la probabilidad asociada, bajo el supuesto de independencia, a todas las tablas que se pueden formar con los mismos totales marginales que los datos observados y variando las frecuencias de cada casilla para contemplar todas las situaciones en las que hay un desequilibrio de proporciones tan grande o más que en la tabla analizada. Para el cálculo de la probabilidad asociada a cada tabla se utiliza la función de probabilidad de una variable discreta hipergeométrica.

Aunque generalmente el test Exacto de Fisher es más conservador que la Chi cuadrado (resulta más complicado que detecte diferencias estadísticamente significativas entre las proporciones), no obstante tiene la ventaja de que se puede aplicar sin ninguna restricción en las frecuencias de las casillas de la tabla de contingencia.


\subsection{Test de McNemar para datos emparejados}
Hasta ahora hemos supuesto que las muestras a comparar eran independientes, es decir dos grupos diferentes en los que se había mirado una
determinada característica. Por lo tanto, hemos realizado comparaciones de proporciones de individuos que presentan una determinada
característica en dos grupos distintos, pero también nos podemos plantear comparar la proporción de individuos que presentan esa
característica en un mismo grupo de individuos pero analizados en dos momentos diferentes. En este último caso se habla comparación de
proporciones en datos emparejados, pareados o apareados.

Por ejemplo, si queremos ver si existen o no diferencias en la mejora de los síntomas de una determinada enfermedad, y para ello aplicamos
dos fármacos distintos a un grupo de individuos en dos momentos diferentes en los que hayan contraído la misma enfermedad. En este caso,
podría pensarse que resultaría adecuado aplicar tanto la chi cuadrado como el test exacto de Fisher para determinar si existe diferencias
entre ambos fármacos en la proporción de pacientes curados, pero aquí hay una diferencia fundamental con los casos anteriores y es que sólo
tenemos un grupo de pacientes y no dos. En este tipo de estudios se reduce considerablemente la variabilidad aleatoria, ya que es un mismo
individuo el que se somete a los dos tratamientos, y el que manifieste mejoría en los síntomas no dependerá de otros factores tan
importantes como, por ejemplo, la edad, el sexo o el tipo de alimentación, que pueden influir pero que tal vez no se controlen adecuadamente
en un diseño de grupos independientes. Al reducir la variabilidad aleatoria mediante datos emparejados, pequeñas diferencias entre las
proporciones pueden llegar a ser significativas, incluso con tamaños muestrales pequeños, lo cual se traduce en que este tipo de diseños del
experimento resultan más eficientes a la hora de obtener resultados estadísticamente significativos.

No obstante, nuevos diseños implican nuevas formas de tratar los datos, y el procedimiento más adecuado es el que se utiliza en el test de
McNemar para datos emparejados. Para su aplicación en nuestro ejemplo, se debería construir una tabla con 4 casillas en las que se
contabilicen: las personas que han obtenido una mejoría de los síntomas con los dos fármacos, los que han obtenido con el primero y no con
el segundo, los que han obtenido con el segundo y no con el primero y los que no han obtenido mejoría con ninguno.

\[
\begin{array}{|c|c|c|c|}
\hline
\textrm{Mejoría 1º $\setminus$ Mejoría 2º} & \textrm{Sí} & \textrm{No} & \textrm{Totales} \\
\hline
\textrm{Sí} & a & b & a+b \\
\hline
\textrm{No} & c & d & c+d \\
\hline
\textrm{Totales} & a+c & b+d & n=a+b+c+d \\
\hline
\end{array}
\]

Con ello, la proporción muestral de pacientes que han experimentado mejoría con el medicamento 1 vale: $\widehat{p}_1=(a+b)/n$, e igualmente
con el 2: $\widehat{p}_2=(a+c)/n$, y podemos plantear el contraste cuya hipótesis nula es que no hay diferencia de proporciones
poblacionales entre ambos medicamentos: $H_0: p_1=p_2$, que puede realizarse sin más que tener en cuenta el oportuno intervalo de confianza
para la diferencia de proporciones, o también que, en el supuesto de igualdad de proporciones
\[z = \dfrac{b-c}{\sqrt{b + c}},\]
es un estadístico que sigue una distribución normal tipificada, y 
\[\chi ^2 = \dfrac{(b-c)^2}{b+c}$,\]
es un estadístico que sigue una distribución Chi-cuadrado con un grado de libertad. 
Con cualquiera de ellos, se podría calcular el p-valor del contraste.

\clearpage
\newpage

\section{Ejercicios Resueltos}
\begin{enumerate}[leftmargin=*]
\item Dadas dos parejas de genes Aa y Bb, la descendencia del
cruce efectuado según las leyes de Mendel, debe estar compuesto
del siguiente modo:

\[
\begin{tabular}{ll}
\multicolumn{1}{c}{Fenotipo} & \multicolumn{1}{c}{Frecuencias Relativas} \\
\multicolumn{1}{c}{AB} & \multicolumn{1}{c}{9/16 = 0,5625} \\
\multicolumn{1}{c}{Ab} & \multicolumn{1}{c}{3/16 = 0,1875} \\
\multicolumn{1}{c}{aB} & \multicolumn{1}{c}{3/16 = 0,1875} \\
\multicolumn{1}{c}{ab} & \multicolumn{1}{c}{1/16 = 0,0625} \\
\end{tabular}
\]

Elegidos 300 individuos al azar de cierta población, se observa la
siguiente distribución de frecuencias:

\[
\begin{tabular}{ll}
\multicolumn{1}{c}{Fenotipo} & \multicolumn{1}{c}{Frecuencias Observadas} \\
\multicolumn{1}{c}{AB} & \multicolumn{1}{c}{165} \\
\multicolumn{1}{c}{Ab} & \multicolumn{1}{c}{47} \\
\multicolumn{1}{c}{aB} & \multicolumn{1}{c}{67} \\
\multicolumn{1}{c}{ab} & \multicolumn{1}{c}{21} \\
\end{tabular}
\]

Se pide
\begin{enumerate}
\item Crear las variables \textsf{fenotipo} y \textsf{frecuencia} e introducir los datos de la muestra.
\item Ponderar los datos mediante la variable \variable{frecuencia}.
\begin{indicacion}
\begin{enumerate}
\item Seleccionar el menú \menu{Datos\flecha Ponderar casos}.
\item En el cuadro de diálogo resultante activar la opción \opcion{Ponderar casos
mediante}, seleccionar la variable \variable{frecuencia} en el campo \campo{Variable frecuencia} y hacer click en el botón \boton{Aceptar}.
\end{enumerate}
\end{indicacion}


\item Comprobar si esta muestra cumple las leyes de Mendel.
\begin{indicacion}
\begin{enumerate}
\item Seleccionar el men\'{u} \texttt{Analizar\flecha Pruebas no paramétricas\flecha Cuadro de diálogo antiguos\flecha Chi-cuadrado}.
\item En el cuadro de diálogo que aparece seleccionar la variable \variable{fenotipo} al campo \campo{Lista Contrastar Variables}, y en \texttt{Valores esperados} marcar la opción \opcion{valores} e
introducir las proporciones según las leyes de Mendel y siguiendo
el orden en el que aparecen los fenotipos, y hacer click sobre el botón \boton{Aceptar}.
\end{enumerate}
\end{indicacion}

\item A la vista de los resultados del contraste, ¿se puede aceptar que se cumplen las leyes de Mendel en los individuos de dicha población?
\end{enumerate}


\item En un estudio sobre úlceras pépticas se determinó el grupo
sanguíneo de 1655 pacientes ulcerosos y 10000 controles, los datos
fueron:

\[
\begin{tabular}{|l|l|l|l|l|}
\cline{2-5}
\multicolumn{1}{c|}{} & \multicolumn{1}{c|}{O} & \multicolumn{1}{c|}{A} & \multicolumn{1}{c|}{B} & \multicolumn{1}{c|}{AB} \\
\hline
\multicolumn{1}{|c|}{Paciente} & \multicolumn{1}{c|}{911} & \multicolumn{1}{c|}{579} & \multicolumn{1}{c|}{124} & \multicolumn{1}{c|}{41} \\
\hline
\multicolumn{1}{|c|}{Controles} & \multicolumn{1}{c|}{4578} & \multicolumn{1}{c|}{4219} & \multicolumn{1}{c|}{890} & \multicolumn{1}{c|}{313} \\
\hline
\end{tabular}
\]

\begin{enumerate}
\item Crear las variables \textsf{participantes}, \textsf{grupo\_sanguíneo} y \textsf{frecuencia} e introducir los datos.
\item Ponderar los datos mediante la variable \variable{frecuencia}.
\begin{indicacion}
\begin{enumerate}
\item Seleccionar el menú \menu{Datos\flecha Ponderar casos}.
\item En el cuadro de diálogo resultante activar la opción \opcion{Ponderar casos
mediante}, seleccionar la variable \variable{frecuencia} en el campo \campo{Variable frecuencia} y hacer click en el botón \boton{Aceptar}.
\end{enumerate}
\end{indicacion}

\item Construir la tabla de contingencia y realizar el contraste Chi-cuadrado.
\begin{indicacion}
\begin{enumerate}
\item Seleccionar el menú \menu{Analizar\flecha Estadísticos Descriptivos\flecha Tablas de
contingencia}.
\item En el cuadro de diálogo que aparece, seleccionar la variable \variable{participantes} al campo \campo{Filas} y la variable \texttt{grupo\_sanguíneo} al campo \campo{Columnas}, y hacer click sobre el botón \boton{Estadísticos}.
\item En el cuadro de diálogo que aparece, marcar la casilla \texttt{Chi-cuadrado} y hacer click en el botón \boton{Continuar}.
\item En el cuadro de diálogo inicial, hacer click sobre el botón \boton{Casillas}.
\item En el cuadro de diálogo que aparece, marcar las casillas \texttt{Frecuencias observadas} y \texttt{Esperadas}, y hacer click sobre el botón \boton{Continuar} y \texttt{Aceptar}.
\end{enumerate}
\end{indicacion}

\item A la vista de los resultados del contraste, ¿existe alguna relación entre el grupo sanguíneo y la úlcera
péptica?, es decir, ¿se puede concluir que la proporción de
pacientes y de controles es diferente dependiendo del grupo
sanguíneo?
\end{enumerate}

\item Mitchell et al. (1976, Annals of Human Biology), partiendo
de una muestra de 478 individuos, estudiaron la distribución de
los grupos sanguíneos en varias regiones del sur-oeste de Escocia,
obteniendo los resultados que se muestran:
\[
\begin{tabular}{|l|l|l|l|l|}
\cline{2-4}
\multicolumn{1}{c|}{} & \multicolumn{1}{c|}{Eskdale} & \multicolumn{1}{c|}{Annandale} & \multicolumn{1}{c|}{Nithsdale} & \multicolumn{1}{c}{} \\
\hline
\multicolumn{1}{|c|}{A} & \multicolumn{1}{c|}{33} & \multicolumn{1}{c|}{54} & \multicolumn{1}{c|}{98} & \multicolumn{1}{c|}{185} \\
\hline
\multicolumn{1}{|c|}{B} & \multicolumn{1}{c|}{6} & \multicolumn{1}{c|}{14} & \multicolumn{1}{c|}{35} & \multicolumn{1}{c|}{55} \\
\hline
\multicolumn{1}{|c|}{O} & \multicolumn{1}{c|}{56} & \multicolumn{1}{c|}{52} & \multicolumn{1}{c|}{115} & \multicolumn{1}{c|}{223} \\
\hline
\multicolumn{1}{|c|}{AB} & \multicolumn{1}{c|}{5} & \multicolumn{1}{c|}{5} & \multicolumn{1}{c|}{5} & \multicolumn{1}{c|}{15} \\
\hline
\multicolumn{1}{c|}{} & \multicolumn{1}{c|}{100} & \multicolumn{1}{c|}{125} & \multicolumn{1}{c|}{253} & \multicolumn{1}{c|}{478} \\
\cline{2-5}
\end{tabular}
\]

\begin{enumerate}
\item Crear las variables \textsf{grupo\_sanguíneo},
\textsf{región} y \textsf{frecuencia} e introducir los datos.
\item Ponderar el estudio, por la variable \variable{frecuencia}
\begin{indicacion}
\begin{enumerate}
\item Seleccionar el menú \menu{Datos\flecha Ponderar casos}.
\item En el cuadro de diálogo resultante activar la opción \opcion{Ponderar casos
mediante}, seleccionar la variable \variable{frecuencia} en el campo \campo{Variable frecuencia} y hacer click en el botón \boton{Aceptar}.
\end{enumerate}
\end{indicacion}

\item Construir la tabla de contingencia y realizar el contraste Chi-cuadrado.
\begin{indicacion}
\begin{enumerate}
\item Seleccionar el menú \menu{Analizar\flecha Estadísticos Descriptivos\flecha Tablas de
contingencia}.
\item En el cuadro de diálogo que aparece, seleccionar la variable \variable{grupo\_sanguíneo} al campo \campo{Filas} y la variable \texttt{región} al campo \campo{Columnas}, y hacer click sobre el botón \boton{Estadísticos}.
\item En el cuadro de diálogo que aparece, marcar la casilla \texttt{Chi-cuadrado} y hacer click en el botón \boton{Continuar}.
\item En el cuadro de diálogo inicial, hacer click sobre el botón \boton{Casillas}.
\item En el cuadro de diálogo que aparece, marcar las casillas \texttt{Frecuencias observadas} y \texttt{Esperadas}, y hacer click sobre el botón \boton{Continuar} y \texttt{Aceptar}.
\end{enumerate}
\end{indicacion}

\item En vista de los resultados del contraste, ¿se distribuyen los grupos sanguíneos de igual manera en las
diferentes regiones?
\end{enumerate}

\item En un estudio para saber si el habito de fumar está relacionado con el sexo, se ha preguntado a 26 personas. De los 9 hombres consultados 2 respondieron que fumaban, mientras que de las 17 mujeres consultadas, 6 fumaban. ¿Podemos afirmar que existe relación entre ambas variables?
\begin{enumerate}
\item Crear las variables \textsf{sexo},
\textsf{fuma} y \textsf{frecuencia} e introducir los datos.
\item Ponderar el estudio, por la variable \variable{frecuencia}
\begin{indicacion}
\begin{enumerate}
\item Seleccionar el menú \menu{Datos\flecha Ponderar casos}.
\item En el cuadro de diálogo resultante activar la opción \opcion{Ponderar casos
mediante}, seleccionar la variable \variable{frecuencia} en el campo \campo{Variable frecuencia} y hacer click en el botón \boton{Aceptar}.
\end{enumerate}
\end{indicacion}

\item Construir la tabla de contingencia y realizar el contraste Chi-cuadrado.
\begin{indicacion}
\begin{enumerate}
\item Seleccionar el menú \menu{Analizar\flecha Estadísticos Descriptivos\flecha Tablas de
contingencia}.
\item En el cuadro de diálogo que aparece, seleccionar la variable \variable{sexo} al campo \campo{Filas} y la variable \texttt{fuma} al campo \campo{Columnas}, y hacer click sobre el botón \boton{Estadísticos}.
\item En el cuadro de diálogo que aparece, marcar la casilla \texttt{Chi-cuadrado} y hacer click en el botón \boton{Continuar}.
\item En el cuadro de diálogo inicial, hacer click sobre el botón \boton{Casillas}.
\item En el cuadro de diálogo que aparece, marcar las casillas \texttt{Frecuencias observadas} y \texttt{Esperadas}, y hacer click sobre el botón \boton{Continuar} y \texttt{Aceptar}.
\end{enumerate}
\end{indicacion}

\item En vista de los resultados del contraste, ¿se distribuyen los fumadores de igual manera en ambos sexos?

\begin{indicacion}
En este caso el procedimiento a seguir es igual que para la Chi cuadrado, pero vemos que ahora no se cumplen las condiciones para poder aplicar esta prueba, por eso nos tendremos que fijar en el \textsf{Estadístico exacto de Fisher}, que si podemos aplicar, teniendo en cuenta si estamos realizando un contraste bilateral o unilateral.
\end{indicacion}
\end{enumerate}

\item Para probar la eficacia de dos fármacos diferentes contra las migrañas, se seleccionaron a 20 personas que padecían migrañas
habitualmente, y se les dió a tomar a cada uno los fármacos en momentos diferentes. Luego se les preguntó si habían obtenido mejoría o no
con el fármaco tomado. Los resultados fueron los siguientes:
\[
\begin{tabular}{|l|l|l|l|l|l|l|l|l|l|l|}
\cline{2-11}
\multicolumn{1}{c|}{} & \multicolumn{1}{c|}{1} & \multicolumn{1}{c|}{2} & \multicolumn{1}{c|}{3} & \multicolumn{1}{c|}{4}& \multicolumn{1}{c|}{5} & \multicolumn{1}{c|}{6} & \multicolumn{1}{c|}{7} & \multicolumn{1}{c|}{8}& \multicolumn{1}{c|}{9} & \multicolumn{1}{c|}{10}  \\
\hline
\multicolumn{1}{|c|}{Fármaco 1} & \multicolumn{1}{c|}{Sí} & \multicolumn{1}{c|}{Sí} & \multicolumn{1}{c|}{Sí} & \multicolumn{1}{c|}{Sí}& \multicolumn{1}{c|}{Sí} & \multicolumn{1}{c|}{No} & \multicolumn{1}{c|}{Sí} & \multicolumn{1}{c|}{No}& \multicolumn{1}{c|}{Sí} & \multicolumn{1}{c|}{Sí}  \\
\hline
\multicolumn{1}{|c|}{Fármaco 2} & \multicolumn{1}{c|}{No} & \multicolumn{1}{c|}{No} & \multicolumn{1}{c|}{Sí} & \multicolumn{1}{c|}{No} & \multicolumn{1}{c|}{Sí} & \multicolumn{1}{c|}{Sí} & \multicolumn{1}{c|}{No} & \multicolumn{1}{c|}{No}& \multicolumn{1}{c|}{No} & \multicolumn{1}{c|}{No} \\
\hline
\end{tabular}
\]
 \[
\begin{tabular}{|l|l|l|l|l|l|l|l|l|l|l|l|}
\cline{2-11}
\multicolumn{1}{c|}{}  & \multicolumn{1}{c|}{11} & \multicolumn{1}{c|}{12}& \multicolumn{1}{c|}{13} & \multicolumn{1}{c|}{14} & \multicolumn{1}{c|}{15} & \multicolumn{1}{c|}{16}& \multicolumn{1}{c|}{17} & \multicolumn{1}{c|}{18} & \multicolumn{1}{c|}{19} & \multicolumn{1}{c|}{20} \\
\hline
\multicolumn{1}{|c|}{Fármaco 1} & \multicolumn{1}{c|}{Sí} & \multicolumn{1}{c|}{No}& \multicolumn{1}{c|}{Sí} & \multicolumn{1}{c|}{No} & \multicolumn{1}{c|}{Sí} & \multicolumn{1}{c|}{Sí}& \multicolumn{1}{c|}{Sí} & \multicolumn{1}{c|}{No} & \multicolumn{1}{c|}{Sí} & \multicolumn{1}{c|}{Sí} \\
\hline
\multicolumn{1}{|c|}{Fármaco 2} & \multicolumn{1}{c|}{Sí} & \multicolumn{1}{c|}{No}& \multicolumn{1}{c|}{Sí} & \multicolumn{1}{c|}{No} & \multicolumn{1}{c|}{No} & \multicolumn{1}{c|}{Sí}& \multicolumn{1}{c|}{No} & \multicolumn{1}{c|}{Sí} & \multicolumn{1}{c|}{No} & \multicolumn{1}{c|}{No}\\
\hline
\end{tabular}
\]

\begin{enumerate}
\item Crear las variables \textsf{Mejora\_Farmaco1},
 y \textsf{Mejora\_Farmaco2} e introducir los datos.
\item Construir la tabla de contingencia y realizar el contraste de McNemar.
\begin{indicacion}
\begin{enumerate}
\item Seleccionar el menú \menu{Analizar\flecha Estadísticos Descriptivos\flecha Tablas de
contingencia}.
\item En el cuadro de diálogo que aparece, seleccionar la variable \variable{Mejora\_Farmaco1} al campo \campo{Filas} y la variable \texttt{Mejora\_Farmaco2} al campo \campo{Columnas}, y hacer click sobre el botón \boton{Estadísticos}.
\item En el cuadro de diálogo que aparece, marcar la casilla \texttt{McNemar} y hacer click en el botón \boton{Continuar}.
\item En el cuadro de diálogo inicial, hacer click sobre el botón \boton{Casillas}.
\item En el cuadro de diálogo que aparece, marcar las casillas \texttt{Frecuencias observadas} y \texttt{Esperadas}, y hacer click sobre el botón \boton{Continuar} y \texttt{Aceptar}.
\end{enumerate}
\end{indicacion}

\item En vista de los resultados del contraste, ¿podemos afirmar que existen diferencias significativas entre los dos fármacos?

\begin{indicacion}
Otra forma de realizar este mismo contraste, sería seleccionado el menú \menu{Analizar\flecha Pruebas no paramétricas\flecha Cuadros de diálogo antiguos\flecha 2 muestras relacionadas}. Luego pasar las dos variables a contrastar al cuadro \texttt{Contrastar pares}, marcar la casilla \texttt{McNemar}y hacer click sobre el botón  \texttt{Aceptar}.
\end{indicacion}
\end{enumerate}

\end{enumerate}


\section{Ejercicios propuestos}
\begin{enumerate}
\item Supongamos que queremos comprobar si un dado está bien equilibrado o no. Lo lanzamos 1200 veces, y obtenemos los siguientes
resultados:
\[
\begin{tabular}{ll}
\multicolumn{1}{c}{Número} & \multicolumn{1}{c}{Frecuencias de aparición} \\
\multicolumn{1}{c}{1} & \multicolumn{1}{c}{120} \\
\multicolumn{1}{c}{2} & \multicolumn{1}{c}{275} \\
\multicolumn{1}{c}{3} & \multicolumn{1}{c}{95} \\
\multicolumn{1}{c}{4} & \multicolumn{1}{c}{310} \\
\multicolumn{1}{c}{5} & \multicolumn{1}{c}{85} \\
\multicolumn{1}{c}{6} & \multicolumn{1}{c}{315} \\
\end{tabular}
\]
\begin{enumerate}
\item A la vista de los resultados, ¿se puede aceptar que el dado está bien equilibrado?
\item Nos dicen que, en este dado, los números pares aparecen con una frecuencias 3 veces superior a la de los impares. Contrastar dicha
hipótesis.
\end{enumerate}

\item Se realiza un estudio en una población de pacientes críticos hipotéticos y se observan, entre otras, dos variables, la evolución (si
sobreviven SV o no NV) y la presencia o ausencia de coma, al ingreso. Se obtienen los siguientes resultados:
\[
\begin{tabular}{l|l|l|l|}
\cline{2-3}
\multicolumn{1}{c|}{} & \multicolumn{1}{c|}{No coma} & \multicolumn{1}{c|}{Coma} & \multicolumn{1}{c}{} \\
\hline
\multicolumn{1}{|c|}{SV} & \multicolumn{1}{c|}{484} & \multicolumn{1}{c|}{37} & \multicolumn{1}{c|}{521} \\
\hline
\multicolumn{1}{|c|}{NV} & \multicolumn{1}{c|}{118} & \multicolumn{1}{c|}{89} & \multicolumn{1}{c|}{207} \\
\hline
\multicolumn{1}{c|}{} & \multicolumn{1}{c|}{602} & \multicolumn{1}{c|}{126} & \multicolumn{1}{c|}{728} \\
\cline{2-4}
\end{tabular}
\]
Nos preguntamos: ¿es el coma al ingreso un factor de riesgo para la mortalidad?

\item La recuperación producida por dos tratamientos distintos A y B, se clasifican en tres categorías: muy buena, buena y mala. Se
administra el tratamiento A a 32 pacientes y el B a otros 28. De las 22 recuperaciones muy buenas, 10 corresponden al tratamiento A; de las
24 recuperaciones buenas, 14 corresponden al tratamiento A y de las 14 que tienen una mala recuperación, 8 corresponden al tratamiento A.
¿Son igualmente efectivos ambos tratamientos para la recuperación de los pacientes?

\item Para contrastar la hipótesis de que las mujeres tienen más éxito en sus estudios que los hombres, se ha tomado una muestra de 10
chicos y otra de 10 chicas que han sido examinados por un profesor que aprueba siempre al 40\% de los alumnos presentados a examen. Teniendo
en cuenta que sólo aprobaron 2 chicos, utiliza el test de hipótesis más adecuado para decidir si la citada hipótesis es cierta.

\item Se ha preguntado a los 150 alumnos de un curso, si estaban de acuerdo o no, con la metodología de enseñanza de dos profesores,
distintos que les han dado clase en la asignatura de bioestadística. Los resultados se recogen en la siguiente tabla:
\begin{center}
\begin{tabular}{|l|c|c|}
\cline{2-3}
\multicolumn{1}{c|}{Profesor 1 $\backslash$ Profesor 2} & Opinión favorable & Opinión desfavorable  \\
\hline
Opinión favorable & 37 & 48  \\
\hline
Opinión desfavorable & 44 & 21 \\
\hline
\end{tabular}
\end{center}

¿Podemos afirmar que existe diferente opinión por parte de los alumnos, sobre los dos profesores?

\end{enumerate}

